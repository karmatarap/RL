\documentclass[12pt]{letter}
\usepackage[english]{babel}
\usepackage{amsmath}
%\pagestyle{empty}
\textwidth=6.1in \textheight=24cm \voffset=-3.7cm
\oddsidemargin=3.6mm
\usepackage{graphicx}
\pagestyle{empty}

\usepackage{enumerate}

\newcommand{\E}{\mathrm{E}}
\newcommand{\Var}{\mathrm{Var}}
\newcommand{\Cov}{\mathrm{Cov}}
\newcommand{\Corr}{\mathrm{Corr}}


\begin{document}
\begin{flushleft}
{\sc Name: \ldots\ldots\ldots\ldots\ldots\ldots\ldots\ldots\ldots\ldots\ldots\ldots\ldots}\\
CSCI E-89c Deep Reinforcement Learning\\
Part I of Assignment 2\\
\end{flushleft}

Please consider a Markov Decision Process with two states: $s^{A}$ and $s^{B}$.\medskip\\
Assume that the sets of admissible actions in states $s^A$ and $s^B$ are $\mathcal{A}(s^A)=\{a_1^A,a_2^A\}$ and $\mathcal{A}(s^B)=\{a_1^B,a_2^B \}$, respectively. Further, assume that the transition probabilities are given by:
\small{\begin{equation*}
\begin{aligned}
p(s^\prime, r|s^A,a_1^A)=&\begin{cases}
1,& \text{ if } s^\prime=s^A, r=r_1^A,\\
0,& \text{ otherwise},
\end{cases}\\
p(s^\prime, r|s^A,a_2^A)=&\begin{cases}
1,& \text{ if } s^\prime=s^A, r=r_2^A,\\
0,& \text{ otherwise},
\end{cases}\\
p(s^\prime, r|s^B,a_1^B)=&\begin{cases}
1,& \text{ if } s^\prime=s^B, r=r_1^B,\\
0,& \text{ otherwise},
\end{cases}\\
p(s^\prime, r|s^B,a_2^B)=&\begin{cases}
1,& \text{ if } s^\prime=s^B, r=r_2^B,\\
0,& \text{ otherwise},
\end{cases}
\end{aligned}
\end{equation*}}where $r_1^A$, $r_2^A$, $r_1^B$, and $r_2^B$ are known.\medskip\\
If policy $\pi(a|s)$ is to always take action $a_1^A$ in state $s^A$ and action $a_1^B$ in state $s^B$, find
\begin{enumerate}[(a)]
\item $v_\pi(s^A)$
\item $q_\pi(s^A,a_1^A)$
\item $q_\pi(s^A,a_2^A)$
\end{enumerate}
SOLUTION:

(a) Policy is to always take action $a_1^A$ in state $s^A$ with reward $r_1^A$ 

$v_\pi(s^A) \doteq r_1^A + \gamma  r_1^A  + \gamma^2 r_1^A + ... \gamma^{n-1} r_1^A = \sum_{k=0}^\infty \gamma^k r_1^A = \frac{r_1^A}{1 - \gamma}$


(b) Action value of $q_\pi(s^A,a_1^A) \doteq v_\pi(s^A)$

Therefore:
$q_\pi(s^A,a_1^A) = \frac{r_1^A}{1 - \gamma}$ also


(c) $q_\pi(s^A,a_2^A) = r_2^A + \gamma  r_1^A  + \gamma^2 r_1^A + ... \gamma^{n-1} r_1^A = \gamma \frac{r_1^A}{1 - \gamma}$



\end{document}
