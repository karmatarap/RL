\documentclass[12pt]{letter}
\usepackage[english]{babel}
\usepackage{amsmath}
%\pagestyle{empty}
\textwidth=6.1in \textheight=24cm \voffset=-3.7cm
\oddsidemargin=3.6mm
\usepackage{graphicx}
\pagestyle{empty}

\usepackage{enumerate}

\newcommand{\E}{\mathrm{E}}
\newcommand{\Var}{\mathrm{Var}}
\newcommand{\Cov}{\mathrm{Cov}}
\newcommand{\Corr}{\mathrm{Corr}}


\begin{document}
\begin{flushleft}
{\sc Name: Karma Tarap}\\
CSCI E-89c Deep Reinforcement Learning\\
Part I of Assignment 5\\
\end{flushleft}

Please consider a Markov Decision Process with  $\mathcal{S}=\{s^{A},s^{B},s^{C}\}$.\medskip\\
Given a particular state $s\in \mathcal{S}$, the agent is allowed to either try staying there or switching to one of the ``neighboring'' states. Let's denote an intention to stay by $0$, an attempt to move to the left by $-1$, and an intention to move to the right by $+1$. The agent does not know transition probabilities, including the distributions of rewards.\medskip\\
Suppose the agent uses the following behavior policy $b(a|s)$:
\small{\begin{equation*}
\begin{aligned}
b(a|s^A)=&\begin{cases}
0.5,& \text{ if } a=0,\\
0.5,& \text{ if } a=1,
\end{cases}\\
b(a|s^B)=&\begin{cases}
1/3,& \text{ if } a=-1,\\
1/3,& \text{ if } a=0,\\
1/3,& \text{ if } a=+1,
\end{cases}\\
b(a|s^C)=&\begin{cases}
0.5,& \text{ if } a=-1,\\
0.5,& \text{ if } a=0,
\end{cases}
\end{aligned}
\end{equation*}}to generate two episodes:\medskip\\
{\small
{\it episode 1:}\\
$S_0=s^A,A_0=+1,R_1=40, S_1=s^B,A_1=+1,R_2=30, S_2=s^C,A_2=0,R_3=70$;\medskip\\
{\it episode 2:}\\
$S_0=s^C,A_0=-1,R_1=10, S_1=s^B,A_1=+1,R_2=50, S_2=s^C,A_2=0,R_3=10$.\medskip\\}
Using the First-Visit Monte Carlo (MC) prediction algorithm for estimating $v_\pi(s)$, please estimate 
\begin{enumerate}[(a)]
\item $v_\pi(s^A)$,
\item $v_\pi(s^B)$,
\item $v_\pi(s^C)$,
\end{enumerate}
where the target policy is $\pi(+1|s)=1$ for $s\in\{s^A,s^B\}$ and $\pi(0|s^C)=1$. Assume $\gamma=0.9$.\medskip\\
{\bf Hint:} Find the importance-sampling ratios $\rho_{t:(3-1)}$ for both episodes in cases (a)--(c).\medskip\\
SOLUTION:
(a)
$v_\pi(s^A)$ = $R_1 * \frac{\pi(1|s^A)}{b(1|s^A)} + \gamma * R_2 * \frac{\pi(1|s^B)}{b(1|s^B)} + \gamma^2 * R_3 * \frac{\pi(0|s^C)}{b(0|s^C)}$

= $R_1 * \frac{\pi(1|s^A)}{b(1|s^A)} + \gamma * R_2 * \frac{\pi(1|s^B)}{b(1|s^B)} + \gamma^2 * R_3 * \frac{\pi(0|s^C)}{b(0|s^C)}$

= $40 * \frac{1}{0.5} + 0.9 * 30 * \frac{1}{1/3} + 0.9^2 70 * \frac{1}{0.5}$

= 274.4

(b)
$v_\pi(s^B)$ = $\frac{(R_2 * \frac{\pi(1|s^B)}{b(1|s^B)} + \gamma * R_3 * \frac{\pi(0|s^C)}{b(0|s^C)}) + (R_2 * \frac{\pi(1|s^B)}{b(1|s^B)} + \gamma * R_3 * \frac{\pi(0|s^C)}{b(0|s^C)})}{2}$

= $\frac{(30 * \frac{1}{1/3} + 0.9 * 70 * \frac{1}{0.5}) + (50 * \frac{1}{1/3} + 0.9 * 10 * \frac{1}{0.5})}{2}$
= 192

(c)
$v_\pi(s^C)$ = $\frac{(70 * 2) + (10*0 + 50 * 0.9 * 3 + 10 * 0.9^2 * 2)}{2}$
$\frac{140 + 151.2}{2} = 145.6$


\end{document}
